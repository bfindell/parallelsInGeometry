
\newpage

\setcounter{chapter}{3}
\chapter{Toward Congruence and Similarity}

\setcounter{section}{3}
\section{Dilations, Scaling, and Similarity (cont.)}
These pages are to follow p. 94 in the 2015 version of the notes for Math 1166.  


\subsection{Theorems for Similar Triangles}

Recall the following:  
\begin{definition}
A geometric figure is similar to another if the second can be obtained
from the first by a sequence of rotations, reflections, translations, and dilations.
\end{definition}

We need to show that this general definition of similarity fits with our previous ideas about similar triangles.  Here is one way of thinking about similar triangles:  

\[
\tri ABC \sim \tri A'B'C' \qquad\Leftrightarrow \qquad\begin{array}{l}
\angle A \simeq \angle A'\\
\angle B \simeq \angle B' \\
\angle C \simeq \angle C'
\end{array}
\]

\begin{question} 
What does this mean?  
\end{question}
\QM

Here is another way of thinking about similar triangles:  
\[
\tri ABC \sim \tri A'B'C' \qquad\Leftrightarrow \qquad
\begin{array}{l}
AB = k\cdot A'B'\\
BC = k\cdot B'C' \\
CA = k\cdot C'A'
\end{array}
\]

\begin{question} 
What does this mean?  
\end{question}
\QM


Using merely the formula for the area of a triangle, we (meaning you)
will explain why the following important theorem is true. Throughout this 
discussion we will use the convention that when we
write $AB$ we mean the \textit{length} of the segment $AB$.


\begin{theorem}[Parallel-Side] 
\index{Parallel-Side Theorem}\index{Theorem!Parallel-Side}
Given:
\[
\includegraphics{../graphics/split1.pdf}
\]
If side $BC$ is parallel to side $DE$, then
\[
\frac{AB}{AD} = \frac{AC}{AE}.
\] 
\end{theorem}

\begin{question}
Can you tell me in English what this theorem says?  How does it relate to the definition of similarity in terms of rigid motions and dilations?  
\end{question}
\QM

Now we (meaning you) are going to explore a bit. See if answering
these questions sheds light on this.

\begin{question} 
If $h$ is the height of $\tri ABC$, find formulas for the areas of
$\tri ABC$ and $\tri ADC$.
\[
\includegraphics{../graphics/split2.pdf} \qquad \includegraphics{../graphics/split3.pdf}
\]
\end{question}
\QM

\begin{question} 
If $g$ is the height of $\tri ACB$, find formulas for the areas of
$\tri ACB$ and $\tri AEB$.
\[
\includegraphics{../graphics/split4.pdf} \qquad \includegraphics{../graphics/split5.pdf}
\]
\end{question}
\QM


\begin{question} Explain why 
\[
\mathrm{Area}(\tri ABC) = \mathrm{Area}(\tri ACB).
\]
\end{question}
\QM

\begin{question} Explain why 
\[
\mathrm{Area}(\tri CBE) = \mathrm{Area}(\tri CBD).
\]
Big hint: Use the fact that you have two parallel sides! Draw a
picture to help clarify your explanation.
\end{question}
\QM

\begin{question} Explain why 
\[
\mathrm{Area}(\tri ADC)= \mathrm{Area}(\tri AEB).
\]
\end{question}
\QM

\begin{question}
Explain why
\[
\frac{\mathrm{Area}(\tri ABC)}{\mathrm{Area}(\tri ADC)} = \frac{ \mathrm{Area}(\tri ACB)}{\mathrm{Area}(\tri AEB)}
\]
\end{question}
\QM

\begin{question} Compute and simplify both of the following expressions:
\[
\frac{\mathrm{Area}(\tri ABC)}{\mathrm{Area}(\tri ADC)} \qquad\text{and}\qquad\frac{ \mathrm{Area}(\tri ACB)}{\mathrm{Area}(\tri AEB)}
\]
\end{question}
\QM


\begin{question} How can you conclude that: 
\[
\frac{AB}{AD} = \frac{AC}{AE}
\]
\end{question}
\QM

\begin{question} 
Why is it important that line $DE$ is parallel to line $CB$?
\end{question}
\QM


\begin{question} 
Can you sketch out (in words) how the questions above prove the Parallel-Side
Theorem?
\end{question}
\QM

Now comes the moment of truth. 
\begin{question}
Can you use the Parallel-Side Theorem to explain why if you know that
if you have two triangles, $\tri ABC$ and $\tri A'B'C'$ with:
\begin{align*}
\angle A &\simeq \angle A'\\
\angle B &\simeq \angle B' \\
\angle C &\simeq \angle C'
\end{align*}
then we must have that
\begin{align*}
AB &= k\cdot A'B'\\
BC &= k\cdot B'C'\\
CA &= k\cdot C'A'
\end{align*}
\end{question}
\marginnote{These notes do not describe why side $CA$ is also scaled by $k$.  You address that question in the Side-Splitter
Theorem activity.}
\QM


\subsubsection{The Converse}

The converse of the Parallel-Side Theorem states:

\begin{theorem}[Split-Side]\index{Split-Side Theorem}\index{Theorem!Split-Side} 
Given:
\[
\includegraphics{../graphics/split1.pdf}
\]
If side $BC$ intersects (splits) the sides of $\tri ADE$ so that
\[
\frac{AB}{AD} = \frac{AC}{AE},
\] 
then side $BC$ is parallel to side $DE$.
\end{theorem}

%\begin{question} 
%How could you investigate this theorem using any of the construction
%techniques above?
%\end{question}
%\QM

Now we (meaning you) will answer questions in the hope that they will
help us see why the above theorem is true.

\begin{question}
Suppose that you \textbf{doubt} that side $BC$ is parallel to side $DE$. Explain
how to place a point $C'$ on side $AE$ so that side $BC'$ is
parallel to line $DE$. Be sure to sketch the situation(s).
\end{question}
\QM

\begin{question} 
You now have a triangle $\tri ADE$ whose sides are split by a line
$BC'$ such that the line $BC'$ is parallel to line $DE$. What does the
Parallel-Side Theorem have to say about this?
\end{question}
\QM

\begin{question} What can you conclude about points $C$ and $C'$?
\end{question}
\QM


\begin{question} 
What does this tell you about the Split-Side Theorem?
\end{question}
\QM




Let's see if you can put this all together:
\begin{question}
Can you use the Split-Side Theorem to explain why you know that
if you have two triangles, $\tri ABC$ and $\tri A'B'C'$ with:
\begin{align*}
AB &= k\cdot A'B'\\
BC &= k\cdot B'C'\\
CA &= k\cdot C'A'
\end{align*}
then we must have that
\begin{align*}
\angle A &\simeq \angle A'\\
\angle B &\simeq \angle B' \\
\angle C &\simeq \angle C'
\end{align*}
\end{question}
\QM

Putting all of our work above together, we may now say the following:





\begin{theorem}\index{similar triangles} 
Two triangles $\tri ABC$ and $\tri A'B'C'$ are 
\textbf{similar} if either equilvalent condition holds:
\[
\begin{array}{l}
\angle A \simeq \angle A'\\
\angle B \simeq \angle B' \\
\angle C \simeq \angle C'
\end{array}
\qquad\text{or}\qquad
\begin{array}{l}
AB = k\cdot A'B'\\
BC = k\cdot B'C' \\
CA = k\cdot C'A'
\end{array}
\]
\end{theorem}

\begin{question}
How does this theorem connect back to the definition of similarity in terms of rigid motions and dilations? 
\end{question}
\QM

%
%   The following theorem 
%
%\subsubsection{SAS-Similarity Theorem}
%
%
%\begin{theorem}[SAS-Similarity Theorem]
%\index{Theorem!SAS Similarity}\index{SAS-Similarity Theorem} Knowing
%the ratio of the lengths of two sides and the measure of the angle
%between them, determines a triangle up to similarity. In pictures, we have something like:
%\[
%\includegraphics{../graphics/split1.pdf}
%\]
%\[
%\frac{AB}{AC} = \frac{AD}{AE} \qquad\Rightarrow\qquad \tri ABC \simeq \tri ADE.
%\]
%\end{theorem}
%
%\begin{question} What does this mean, ``up to similarity?''
%\end{question}
%\QM

%Let's see if we (meaning you) can get to the bottom of why this
%theorem is true. This time, you're going to produce the
%illustrations. Use folding and tracing why not!
%
%\begin{question}
%Fold any triangle. Now fold another triangle sharing one of the angles
%so that the ratio of the lengths of the sides are the same in both
%triangles. The sides touching the angle should share folds. You should
%see some parallel lines. Which theorem above says that this should
%happen?
%\end{question}
%\QM
%
%\begin{question} 
%What do we know about parallel lines crossing another line?
%\end{question}
%\QM
%
%\begin{question} 
%Can you sketch out (in words) how the questions above prove the SAS-Similarity
%Theorem?
%\end{question}
%\QM


\subsection{A New Meaning of Multiplication}
School mathematics makes sense when concepts have \emph{meaning}.  

\begin{question} 
What can multiplication mean?  Can you give multiplication meaning involving groups of groups or
something of the sort?
\end{question}
\QM

\begin{question} 
Can you give multiplication meaning involving areas or something of the sort?
\end{question}
\QM

\begin{question} 
Can you somehow give meaning to multiplication using similarity?  Use ``scale factor'' or ``scaling'' in your explanation.
\end{question}
\QM

\newpage 

\subsection{Problem Solving with Similarity} 
We now have several ways of thinking more deeply about the naive ``same shape'' notion of similarity (imagined as zooming in and out): same angles; proportional sides; sequence of basic rigid motions and dilation(s).  

Key issue is being able to distinguish situations in which things are similar from those in which when they are not:  (e.g., baby from PowerPoint, types of televisions).  Multiplication is not necessarily scaling--unless it is dilation.  Direct proportion versus not.  

Many real-world problems can be solved using similar triangles or other similar figures.  For example, you can use shadows to compute the height of a flagpole.  And maps, scale drawings, and scale models all involve similarity.  

Students use proportional relationships between corresponding parts of similar figures, distinguishing ``within figure'' ratios from ``across figure'' ratios, relating the latter to the scale factor.\standardhs{G-SRT.2}  When the figures overlap, one challenge is being consistent about part-part versus part-whole ratios.

Students use the definition of similarity to show that any two circles are similar.\standardhs{G-C.1}  They can also see the more surprising result that any two parabolas are similar.  

Similarity turns out to be very useful in right triangles.  First, the altitude to the hypotenuse creates two triangles similar to the first.  Second, among right triangles, similarity requires only one more angle, which leads to right triangle trigonometry.  

\section{Length, Area, and Volume under Scaling}
When two objects are similar, then lengths are related by a scale factor.  What does this mean for other measurements?  Perimeters, areas, volumes, weights, etc.  

\begin{itemize}\itemsep-2pt
\item General considerations of measurement and dimension.  
\item Reason about length, area, and volume in similarity situations.  Rep-tiles.    
\item Use shearing and Cavalieri's principle to reason about area and volume.  
\item Using a grid (and scaling the grid) to reason about areas of general shapes under scaling.  
\item Volume as area of base times height:  Imagine layers of cubic units covering the base.
\item Volume of pyramid:  Three pyramids make a cube.  
\item Volume of cylinder, cone, and sphere.
\item If you know the area of a rectangle, what can you say about its perimeter?  What about more general figures?  
\item If you know the perimeter of a rectangle, what can you say about its area?  What about more general figures? 
\item LeBron problems
\item Fractals?  
\end{itemize}
