\newpage 

\section{Trigonometry Checkup}
\begin{teachingnote}
This activity can be done either as part of similarity or as a preactivity for Circular Trigonometry.  Perhaps recommend a special office hour for this.  
\end{teachingnote}
This activity is intended to remind you of key ideas from high school trigonometry. 

\begin{prob}
What are the ratios of side lengths in a $45^\circ$-$45^\circ$-$90^\circ$ triangle?  Explain where the ratios come from, including why they work for any such triangle, no matter what size.  (Hint: Use the Pythagorean Theorem.)
\end{prob}

\vspace{0.1in}

\begin{prob}
What are the ratios of side lengths in a $30^\circ$-$60^\circ$-$90^\circ$ triangle?  Explain where those the come from.  (Hint: Think of half of an equilateral triangle.)
\end{prob}

\vspace{0.1in}

\begin{prob}
Consider the right triangle below with an angle of $\alpha$, sides of length $x$ and $y$, and hypotenuse of length $r$, as labeled.  
$$\includegraphics[scale=0.8]{../graphics/rightTriangle}$$
\begin{enumerate}
\itemsep 12pt
\item If we imagine angle $\alpha$ is fixed, why are ratios of pairs of side lengths the same, no matter the size of the triangle?\standardhs{G-SRT.6}
\item Using the triangle above (and your memory of Precalculus), write down the side-length ratios for sine, cosine, and tangent:  
$$\sin\alpha = \hspace{1in} \cos\alpha = \hspace{1in} \tan\alpha =$$
\item What values of $\alpha$ make sense in \emph{right triangle trigonometry}?  (We overcome these bounds later in circular trigonometry.)  
\item What does it mean to say that these ratios depend upon the angle $\alpha$?  
\item Why is only one of the triangle's three angles necessary in defining these ratios?  
\end{enumerate}
\end{prob}

\begin{prob}
Use your work so far to find the following trigonometric ratios:
\begin{enumerate}
\itemsep 12pt
\item $\sin 30^\circ = \hspace{1in} \cos30^\circ = \hspace{1in} \tan30^\circ =$
\item $\sin 45^\circ = \hspace{1in} \cos45^\circ = \hspace{1in} \tan45^\circ =$
\item $\sin 60^\circ = \hspace{1in} \cos60^\circ = \hspace{1in} \tan60^\circ =$
\item $\sin 0^\circ = \hspace{1.08in} \cos0^\circ = \hspace{1.08in} \tan0^\circ =$
\end{enumerate}
\end{prob}

\vspace{0.2in}

\begin{prob}
You may recall the identity $\sin^2\theta+\cos^2\theta=1$.\standardhs{F-TF.8}  
\begin{enumerate}
\item Explain why the equation is true.  
\vspace{0.3in}
\item Why is it called an identity? 
\vspace{0.3in}
\item Why is it called a Pythagorean identity?  
\vspace{0.3in}
\end{enumerate}
\end{prob}

\newpage
\begin{prob}
In right triangle trigonometry, there are indeed two acute angles, as shown in the figure below.\standardhs{G-SRT.7}
$$\includegraphics[scale=0.8]{../graphics/rightTriangle2}$$
\fixnote{Fix the fonts for alpha, beta so they are not different.}
\begin{enumerate}
\item How are the angles $\alpha$ and $\beta$ related?  Explain why.
\vspace{0.3in}
\item Using lengths in the above triangle, find the following ratios:    
$$\sin\alpha = \qquad\qquad\qquad \cos\alpha = $$
$$\sin\beta = \qquad\qquad\qquad \cos\beta = $$
\item What do you notice about the sine and cosine of complementary angles?  
\vspace{0.3in}
\item Explain why the result makes sense.  
\vspace{0.3in}
\end{enumerate}
\end{prob}

Given an angle and a side length of a right triangle, you can find the missing side lengths.\standardhs{G-SRT.8}  This is called ``solving the right triangle.''    And given the sine, cosine, or tangent of an angle, you can find the other two ratios.  (Hint: In either case, draw a triangle.)

%\begin{prob}
%A straight wire to the top of a flagpole meets the ground at a $25^\circ$ angle 30 feet from the base of the flag pole (on a flat lawn).  How high is the flagpole?  How long is the wire?  
%\end{prob}
%
%\vspace{0.5in}

\begin{prob}
Suppose $\sin\alpha = \frac{3}{5}$.  Then $\cos\alpha = \hspace{0.6in}$, $\tan\alpha = \hspace{0.6in}$.  
\end{prob}


