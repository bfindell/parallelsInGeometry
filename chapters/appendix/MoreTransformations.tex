\newpage

\section{More Transformations}
\begin{teachingnote}
Supplies:  tracing paper.  Helpful to label the vertices and pay attention to where they go in the symmetry transformation.  The named lines of symmetry don't move when the figure moves.
\end{teachingnote}

Transformations of the plane are considered to be functions that take points as inputs and produce 
points as outputs.  Given a point as input, the corresponding output value is often called 
the \emph{image} of the point under the transformation.\standardhs{G-CO.2}
\begin{prob}
Based on your experience with the basic rigid motions, write definitions of translation, rotation, and reflection.\standardhs{G-CO.4} For each definition, be sure to indicate (1) what it takes to specify the transformation, and (2) how to produce the image of a given point.  
\begin{enumerate}
\item Translation: 
\vspace{0.3in}
\item Rotation: 
\vspace{0.3in}
\item Reflection: 
\vspace{0.3in}
\end{enumerate}
\end{prob}

\begin{prob}
Now explore sequences of basic rigid motions.  Here are some suggestions to support your explorations:  
\begin{itemize}\itemsep0pt
\item Use a non-symmetric figure (such as an F). 
\item Use one sheet of tracing paper as the original plane, and use a second sheet of paper to carry out the sequence of transformations.  
\item Trace intermediate figures on both sheets of paper, to keep track of the work.   
\item For reflections, trace the line of reflection on both sheets. 
\item For rotations, use a protractor to help you keep track of angles.  
\item Consider special cases, such as reflections about the same line or rotations about the same point.  
\item Try to predict the result before you actually carry out the sequence of transformations.  
\end{itemize}
Describe briefly what you can say about each of the following sequences of basic rigid motions.  Include special cases in your descriptions.  
\begin{enumerate}
\item Translation followed by translation
\vspace{0.5in}
\item Rotation followed by rotation
\vspace{0.5in}
\item Reflection followed by reflection
\vspace{0.5in}
\item Translation followed by rotation
\vspace{0.5in}
\item Translation followed by reflection
\vspace{0.5in}
\item Rotation followed by reflection
\end{enumerate}
\end{prob}

