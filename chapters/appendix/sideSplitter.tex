\newpage

\section{Side-Splitter Theorems}
In this activity, we will show that the properties of dilations, which you noticed in a previous activity, can be proven \emph{without} using facts about transversals and parallel lines.  Instead, we use the area formulas for rectangles, triangles, and parallelograms.  
\begin{question}
What must be true about the base and height measurements for these area formulas to be valid? 
\end{question}

\begin{prob}
If the area of $\triangle SPR = 8$ square inches and the area of $\triangle QPR = 5$ square inches, then what can you say about $\frac{SR}{RQ}$?  What about $\frac{SR}{SQ}$?  What can you say generally about how these ratios depend upon the areas of the triangles?  
$$\includegraphics[scale=0.5]{../graphics/sideSplitter1}$$
\end{prob}

\begin{prob}
For the trapezoid below, explain why the area of $\triangle BAD$ is equal to the area of $\triangle BAC$.  Name two other triangles that have the same area.
$$\includegraphics[scale=0.5]{../graphics/sideSplitter2}$$
\end{prob}

\begin{prob}
For the parallelogram below, which triangle has the greatest area: $\triangle XYZ$, $\triangle WXY$, $\triangle ZWX$, or $\triangle YZW$?  Explain.  
$$\includegraphics[scale=0.5]{../graphics/sideSplitter3}$$
\end{prob}

\begin{teachingnote}
An important objective in the next two problems is the habit of using an equation string, one modification at a time, to show that two expressions are equivalent.
\end{teachingnote}

\begin{prob}
Prove the \textbf{Parallel-Side Theorem}:  If a line in a triangle is parallel to a side of a triangle, then it splits the other sides of the triangle proportionally. 
$$\includegraphics[scale=0.7]{../graphics/sideSplitter}$$
\begin{enumerate}
\item How do the areas of $\triangle ADE$ and $\triangle DBE$ relate to $AD$ and $DB$?  Explain.  
\item How do the areas of $\triangle ADE$ and $\triangle CED$ relate to $AE$ and $CE$?  Explain. 
\item How do the areas of $\triangle BDE$ and $\triangle CED$ compare?  Explain.  
\item Use the previous results to show that $\frac{AD}{DB} = \frac{AE}{EC}$.  
\item Where in the proof did you use the fact that $\overline{DE} \parallel \overline{BC}$?  
\end{enumerate}
\end{prob}

\fixnote{Need a recap:  What the heck did we just do?  What does this say?  Need teaching notes to help set up the algebra to see the structure of the argument.}  

\begin{prob}
Use some algebra to show, in the previous picture, that $\frac{AB}{AD} = \frac{AC}{AE}$.
\end{prob}

\begin{prob}
Prove:  In the previous figure, $ \frac{BC}{DE} = \frac{AB}{AD} = \frac{AC}{AE}$.  
\begin{enumerate}
\item How do we know that $\angle ADE \cong \angle ABC$?  
\item Translate $\triangle ADE$ by the vector $\overrightarrow{DB}$ so that the image $\angle A'D'E'$ of $\angle ADE$ coincides with $\angle ABC$.  Draw a picture of the result.  
\item What segments are parallel now?  How do you know?  
\item Now explain why $\frac{BC}{DE} = \frac{AB}{AD} = \frac{AC}{AE}$ is equal to a common ratio from the previous problem.  
\end{enumerate}
\end{prob}

\begin{prob}
Explain briefly how the Parallel-Side Theorem implies the AA criterion for triangle similarity.  (Hint: Be sure to use the definition of similarity in terms of basic rigid motions and dilations.)  
% Hint:  You need to begin with two triangles that have two pairs of congrent angles.  You need to show (generally) that there exists a sequence of basic rigid motions and dilations that maps one triangle onto the other.  Be sure to consider how you would know whether a reflection is needed as one of the basic rigid motions.  
\end{prob}

\begin{prob}
The \textbf{Split-Side Theorem} is the converse of the Parallel-Side Theorem.   
\begin{enumerate}
\item State the Split-Side Theorem.   
% If a line in a triangle splits two sides proportionally, then it is parallel to the third side of the triangle.
\item Prove the Split-Side Theorem.  (Hint:  Using the previous figures, draw a line through $D$ and parallel to $\overline{BC}$, and let $X$ be the point where the new line intersects $\overline{AC}$.  By the previous results, $\overline{DX}$ divides the sides proportionally.  Then argue that $E$ and $X$ must be the same point.)  
\end{enumerate}
\end{prob}



\begin{prob}
Use the Split-Side Theorem to justify the following properties of a dilation given by a center and a scale factor:
\begin{enumerate}
\item A dilation takes a line not passing through the center of the dilation to a parallel line, and leaves a line passing through the center unchanged.
\item The dilation of a line segment is longer or shorter in the ratio given by the scale factor.
\end{enumerate}
% Hint: You need to begin with a dilation, and you need to show that it has the listed properties.  In addition to the Split-Side Theorem, you may also use other results established in this activity. 

\end{prob}

\begin{prob}
Explain briefly how the Split-Side Theorem establishes the SAS criterion for triangle similarity.  
\end{prob}

