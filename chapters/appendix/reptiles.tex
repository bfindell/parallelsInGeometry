\newpage
\section{Rep-Tiles}

\fixnote{Need to start with some easy ones with parallelograms, triangles, and square numbers.  Also use Delanie's table. Maybe split into two activities.}

A \textbf{rep-tile}\index{rep-tile} is a polygon where several copies of
a given rep-tile fit together to make a larger, similar, version of
itself. If $2$ copies are used, we call it a \textit{rep-2-tile}, if
$3$ copies are used, we call it a \textit{rep-3-tile}, and if $n$ copies
are used, we call it a \textit{rep-n-tile}.


\begin{prob}
With a separate sheet of paper, draw and cut out:
\begin{enumerate}
\item An isosceles right triangle whose sides have lengths $1''$, $1''$, and $\sqrt{2}''$.
\item A rectangle whose sides have lengths $1''$ and $\sqrt{2}''$.
\end{enumerate}
Working with a partner, show that each of these polygons is a rep-2-tile.
\end{prob}

\begin{prob}
For each rep-tile above, compute the perimeter and area. In each case,
how does this relate to the perimeter and area of the larger polygon?
\end{prob}

\begin{teachingnote}
This is at least a two-day activity. Bring printed versions of the figures so that they can cut out already drawn ones. 
Bring a printed version of the table. 
Some time working with the figures, computing areas and perimeters, and practicing arithmetic of radicals.  Move the summary to the second day.
\end{teachingnote}

\begin{prob}
With a fresh sheet of paper, start a table to keep track of your work:  
\begin{center}
\begin{tabular}{c|c|c|c}
rep-tile & scale factor (new:old) &  perimeter (new:old) &  area (new:old)  \\ \hline\hline
\textit{description} &     &      &     \\ 
  $\vdots$    & $\vdots$  &  $\vdots$  &  $\vdots$ \\ 
\end{tabular}
\end{center}
\end{prob}


\begin{prob}
Geometry Giorgio suggests that a rectangle whose sides have lengths
$1''$ and $4''$ is also a rep-2-tile. Is he right? If you should
happen to search the internet for other examples of rep-2-tiles, you
might find a surprise.
\end{prob}


\begin{prob}
With a separate sheet of paper, draw and cut-out:
\begin{enumerate}
\item A 30-60-90 right triangle whose shortest side has length $1''$.
\item A rectangle whose sides have lengths $1''$ and $\sqrt{3}''$.
\end{enumerate}
Working with a partner, show that each of these polygons is a rep-3-tile.
\end{prob}

\begin{prob}
For each rep-tile above, compute the perimeter and area. In each case,
how does this relate to the perimeter and area of the larger polygon?
Add this information to your table.
\end{prob}


\begin{prob}
Explain why every triangle and every parallelogram is a
rep-4-tile. Give an example of each, and compute the perimeter and
area. In both cases, compare the perimeter and area to that of the
larger polygons.
\end{prob}


%\break

\begin{prob}
With a separate sheet of graph paper, draw and cut out the following polygons:
\[
\includegraphics{../graphics/rep-4-tile1.pdf}
\]
Working with a partner, show that each of these polygons is a rep-4-tile.
\end{prob}

\begin{prob}
For each rep-tile above, compute the perimeter and area. In each case,
how does this relate to the perimeter and area of the larger polygon?
Add this information to your table.
\end{prob}


\begin{prob}
With a separate sheet of paper, trace and cut out the following
polygons:
\[
\includegraphics{../graphics/rep-4-tile2.pdf}
\]
Working with a partner, show that each of these polygons is a rep-4-tile.
\end{prob}


\begin{prob}
Explain why every rectangle whose sides have ratio $1:\sqrt{n}$ is a
rep-$n$-tile.
\end{prob}

\begin{prob}
Explain how you know that any rep-tile will tessellate the plane.
\end{prob}

\begin{prob}
Give an example of a polygon that tessellates the plane that is not a
rep-tile.
\end{prob}


\begin{prob}
Every tessellation made by rep-tiles will have \index{symmetry of
scale}\textbf{symmetry of scale}. What does it mean to have \textit{symmetry of scale}?
\end{prob}

\begin{prob}
Consider the tessellations made by rep-tiles you've seen so far. What
other symmetries do they have?
\end{prob}

\begin{prob}
Do you think you can have a tessellation that has symmetry of scale
but no other symmetries?
\end{prob}
