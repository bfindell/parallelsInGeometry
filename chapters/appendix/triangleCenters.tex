\newpage
\section{Triangle Centers}
\begin{teachingnote}
This exploration introduces perpendicular bisectors, angle bisectors, medians, and altitudes and the idea of concurrency.  

Use a Geogebra to demonstrate that two points determines a family of circles.  A third point (usually) specifies the circle.
\end{teachingnote}

In this activity, we use \textsl{Geogebra} to explore the basic lines, centers, and circles related to triangles.  

\begin{prob} Here are some easy questions to get the brain-juices flowing!
\begin{enumerate} 
\itemsep -3pt
\item Place two points randomly in the plane. Do you expect to be able to
draw a single line that connects them?
\item Place three points randomly in the plane. Do you expect to be able to
draw a single line that connects them?
\item Place two lines randomly in the plane. How many points do you expect
them to share?
\item Place three lines randomly in the plane. How many points do you expect
all three lines to share?
\item Place two points randomly in the plane. Will you always be
able to draw a circle containing these points?
\item Place three points randomly in the plane. Will you (almost!) always be
able to draw a circle containing these points? If no, why not? If yes,
how do you know?
\item Place four points randomly in the plane. Do you expect to be able to
draw a circle containing all four at once? Explain your reasoning.
\end{enumerate}
\end{prob}

\begin{definition}
Three (or more) distinct lines are said to be \textbf{concurrent} if they have a point in common.  
\end{definition}

\begin{prob} 
In \textsl{Geogebra}, draw a triangle. Now construct the perpendicular bisectors of
the sides.  Describe what you notice.  Does this work for every triangle?
\end{prob}

\begin{prob}
In a new \textsl{Geogebra} sketch, draw a triangle. Now bisect the angles.  Describe what you notice.  Does this work for every
triangle?
\end{prob}

\begin{prob}
In a new \textsl{Geogebra} sketch, draw a triangle. Now construct the lines containing the altitudes.  Describe what you notice.  
Does this work for every triangle?
\end{prob}

\begin{prob}
In a new \textsl{Geogebra} sketch, draw a triangle. Now construct the medians.  Describe what you notice.  Does this work for every triangle?
\end{prob}

\begin{prob}
The \textbf{circumcircle} of a triangle contains all three vertices of the triangle.  The center of the circumcircle is called the \textbf{circumcenter}.  Find the circumcenter on your sketch with the three perpendicular bisectors, and construct the circumcircle.  
\end{prob}

\begin{prob}
The \textbf{incircle} of a triangle is tangent to all three sides of the triangle.  The center of the incircle is called the \textbf{incenter}.  
Find the incenter on your sketch with three angle bisectors. Construct the incircle.  (Hint:  To find the radius of the incircle, you will need to find the distance from the incenter to one of the sides of the triangle.)  
\end{prob}

\begin{prob}
The other ``centers'' of a triangle are called the \textbf{centroid} and the \textbf{orthocenter}.  Make a thoughtful guess about how these correspond to the medians and the lines containing the altitudes.  
\end{prob}


\begin{prob}
Fill in the following handy chart summarizing what you found above. 
\[
\begin{tabular}{| l || c | c | c |}
\hline
  & Associated point? & \begin{minipage}{14ex}\minipad Always inside the triangle? \minipad\end{minipage} & Meaning? \\ \hline\hline 
\begin{minipage}{12ex}\minipad perpendicular \\ bisectors \minipad\end{minipage}  &\hspace{25mm} &\hspace{15mm}  & \hspace{100mm} \\ \hline
\begin{minipage}{12ex}\minipad angle \\ bisectors \minipad\end{minipage} & \rule[0mm]{0mm}{7mm}    &  & \\ \hline
\begin{minipage}{12ex}\minipad lines \\ containing altitudes \minipad\end{minipage} & \rule[0mm]{0mm}{7mm}    &  &  \\ \hline
\begin{minipage}{12ex}\minipad lines \\ containing the medians  \minipad\end{minipage} & \rule[0mm]{0mm}{7mm}   &  &   \\ \hline
\end{tabular}
\]
Be sure to put this in a safe place like in a safe, or under your bed.
\end{prob}
