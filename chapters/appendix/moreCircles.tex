\newpage

\section{More Circles}

\fixnote{Move some of these to problems.  Make the remaining more ``activity'' like.  Include cutting.}
\begin{prob}
Prove: The radius of a circle is perpendicular to the tangent where the radius intersects the circle.  Hint:  Suppose not. 
\end{prob}

\begin{prob}
Suppose an angle circumscribes a circle, as shown below.  Find a relationship between the measure of the angle and the measure of the central angle intercepted by the same chord.
$$\includegraphics[width=2.5in]{../graphics/circumscribedAngle.pdf}$$
\end{prob}

\begin{prob}
Show that, given any three non-collinear points in the Euclidean
plane, there is a unique circle passing through the three points.
\end{prob}

\begin{prob}
Draw four points in the Euclidean plane, no three of which are collinear, that cannot lie on a single circle.  Explain your reasoning. 
\end{prob}

\begin{prob}
Using a compass, draw a large circle, and inscribe a quadrilateral in the circle.  Measure the four angles.  Repeat with another circle and quadrilateral.  What do you notice?  Identify a condition on any quadrilateral that is inscribed in a circle.  Now prove it.  
\end{prob}

\begin{prob}
Construct a tangent line to a circle from a point outside the given circle.
$$\includegraphics[width=3in]{../graphics/tangent2.pdf}$$
\end{prob}

\begin{prob}
Give an informal derivation of the relationship between the circumference and area of a circle.  Imagine cutting a circle into ``pie pieces'' and rearranging the pieces into a shape like the one below.  As the circle is cut into more and more equal-sized ``pie pieces,'' what does the rearranged shape begin to resemble?  Can you find the area of this shape?  
$$\includegraphics[width=3.5in]{../graphics/circleArea.pdf}$$
\end{prob}

\begin{prob}
Derive a formula for the length of the arc intercepted by an cenral angle of a circle.  
\end{prob}

\begin{prob}
Derive a formula for the area of a sector of a circle.  
\end{prob}

\begin{prob}
Explain the following statement:  Given a central angle of a circle, the length of the intercepted arc is proportional to the radius of the circle.  The radian measure of the angle is the constant of proportionality.  (Hint:  Illustrate your explanation with a familiar angle.)
\end{prob}

