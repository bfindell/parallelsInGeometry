\newpage
\section{Bola, Para Bola}           
\fixnote{Need a better sequence of problems.  (Actually do the algebra to make sure they are okay.)  Include a sideways parabola.  Should we generalize the vertical version?  Somehow explain why it will always yield this form.  Could we have predicted the form before completing all the algebra?  That is the real goal.}
We've mentioned several times that a parabola is the set of points
that are equidistant from a given point (the focus) and a given line
(the directrix):\index{focus}\index{directrix}
\[
\includegraphics[angle=90,scale=.4]{../graphics/parabolapointline.pdf}
\]
In this activity we are going to reconcile the definition given
above with the equation that you know and love (admit it!):
\[
y = ax^2 + bx + c
\]

\begin{prob}
How do we compute the distance between two points? Be explicit!
\end{prob}

\begin{prob}
Let's see if we can derive the formula for a parabola with its focus at $(0,1)$ and its directrix being the line $y=0$.
\begin{enumerate}
\item Given a point $(x,y)$, write an expression for the distance from this point to the focus.
\item Write an expression for the distance from $(x,y)$ to the directrix. 
\item Use these two expressions and some algebra to find the formula for the parabola. 
\end{enumerate}
\end{prob}

\begin{prob}
Let's see if we can derive the formula for a parabola with its focus at $(0,1)$ and its directrix being the line $y=-1$.
\begin{enumerate}
\item Given a point $(x,y)$, write an expression for the distance from this point to the focus.
\item Write an expression for the distance from $(x,y)$ to the directrix. 
\item Use these two expressions and some algebra to find the formula for the parabola. 
\end{enumerate}
\end{prob}


\begin{prob}
Let's see if we can derive the formula for a parabola with its focus at $(1,1)$ and its directrix being the line $y=-2$.
\begin{enumerate}
\item Given a point $(x,y)$, write an expression for the distance from this point to the focus.
\item Write an expression for the distance from $(x,y)$ to the directrix. 
\item Use these two expressions and some algebra to find the formula for the parabola. 
\end{enumerate}
\end{prob}



