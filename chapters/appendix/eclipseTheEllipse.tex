
\newpage
\section{Eclipse the Ellipse}

In this activity we'll investigate parametric plots of ellipses and
other curves.

\begin{prob} 
Recall that for $0\le t<2\pi$ 
\begin{align*}
x(t) &= \cos(t)\\ 
y(t) &= \sin(t)
\end{align*}
gives a parametric plot of a unit circle. Describe the plot of
\begin{align*}
x(t) &= 3\cos(t)\\ 
y(t) &= \sin(t)
\end{align*}
for $0\le t<2\pi$.
\end{prob} 

\begin{prob}
Now describe the plot of 
\begin{align*}
x(t) &= 2\cos(t)\\ 
y(t) &= 5\sin(t)
\end{align*}
for $0\le t<2\pi$.
\end{prob}

\begin{prob}
We claim that an ellipse centered at the origin is defined by points
$(x,y)$ satisfying
\[
\left(\frac{x}{a}\right)^2 + \left(\frac{y}{b}\right)^2 = 1.
\]
Are the parametric curves we found above ellipses? Explain why or why
not. \fixnote{Use this for something.}
\end{prob}

\newpage


\begin{prob} 
Here we have some plots showing two concentric circles and an ellipse that touches both.
\[
\begin{tabular}{ccc}

\begin{tikzpicture}
	\begin{axis}[
            %xmin=-5,xmax=5,ymin=-5,ymax=5,
            width=2.5in,
            clip=false,
            domain=(0:2*pi),
            axis lines=center,
            %ticks=none,
            unit vector ratio*=1 1 1,
            xlabel=$x$, ylabel=$y$,
            every axis y label/.style={at=(current axis.above origin),anchor=south},
            every axis x label/.style={at=(current axis.right of origin),anchor=west},
          ]      
          \pgfmathsetmacro{\a}{0}
          \addplot [very thick, penColor, smooth] ({3*cos(deg(x))},{3*sin(deg(x))});
          \addplot [very thick, penColor, smooth] ({5*cos(deg(x))},{5*sin(deg(x))});
          \addplot [very thick, penColor, smooth] ({5*cos(deg(x))},{3*sin(deg(x))});
          
          \addplot[] plot coordinates {(0,0) ({5*cos(deg(\a))},{5*sin(deg(\a))})}; %% line

          \addplot[color=penColor,fill=penColor,only marks,mark=*] coordinates{({3*cos(deg(\a))},{3*sin(deg(\a))})};  %% closed hole          
          \addplot[color=penColor,fill=penColor,only marks,mark=*] coordinates{({5*cos(deg(\a))},{5*sin(deg(\a))})};  %% closed hole          
          \addplot[color=penColor,fill=penColor,only marks,mark=*] coordinates{({5*cos(deg(\a))},{3*sin(deg(\a))})};  %% closed hole          
        \end{axis}
\end{tikzpicture}
&
\begin{tikzpicture}
	\begin{axis}[
            %xmin=-.1,xmax=1.1,ymin=-.1,ymax=1.1,
            clip=false,
            width=2.5in,
            domain=(0:2*pi),
            axis lines=center,
            %ticks=none,
            unit vector ratio*=1 1 1,
            xlabel=$x$, ylabel=$y$,
            every axis y label/.style={at=(current axis.above origin),anchor=south},
            every axis x label/.style={at=(current axis.right of origin),anchor=west},
          ]      
          \pgfmathsetmacro{\a}{pi/10}
          \addplot [very thick, penColor, smooth] ({3*cos(deg(x))},{3*sin(deg(x))});
          \addplot [very thick, penColor, smooth] ({5*cos(deg(x))},{5*sin(deg(x))});
          \addplot [very thick, penColor, smooth] ({5*cos(deg(x))},{3*sin(deg(x))});
          
          \addplot[] plot coordinates {(0,0) ({5*cos(deg(\a))},{5*sin(deg(\a))})}; %% line

          \addplot[color=penColor,fill=penColor,only marks,mark=*] coordinates{({3*cos(deg(\a))},{3*sin(deg(\a))})};  %% closed hole          
          \addplot[color=penColor,fill=penColor,only marks,mark=*] coordinates{({5*cos(deg(\a))},{5*sin(deg(\a))})};  %% closed hole          
          \addplot[color=penColor,fill=penColor,only marks,mark=*] coordinates{({5*cos(deg(\a))},{3*sin(deg(\a))})};  %% closed hole          
        \end{axis}
\end{tikzpicture}
&
\begin{tikzpicture}
	\begin{axis}[
            %xmin=-.1,xmax=1.1,ymin=-.1,ymax=1.1,
            width=2.5in,
            clip=false,
            domain=(0:2*pi),
            axis lines=center,
            %ticks=none,
            unit vector ratio*=1 1 1,
            xlabel=$x$, ylabel=$y$,
            every axis y label/.style={at=(current axis.above origin),anchor=south},
            every axis x label/.style={at=(current axis.right of origin),anchor=west},
          ]      
          \pgfmathsetmacro{\a}{2*pi/10}
          \addplot [very thick, penColor, smooth] ({3*cos(deg(x))},{3*sin(deg(x))});
          \addplot [very thick, penColor, smooth] ({5*cos(deg(x))},{5*sin(deg(x))});
          \addplot [very thick, penColor, smooth] ({5*cos(deg(x))},{3*sin(deg(x))});
          
          \addplot[] plot coordinates {(0,0) ({5*cos(deg(\a))},{5*sin(deg(\a))})}; %% line

          \addplot[color=penColor,fill=penColor,only marks,mark=*] coordinates{({3*cos(deg(\a))},{3*sin(deg(\a))})};  %% closed hole          
          \addplot[color=penColor,fill=penColor,only marks,mark=*] coordinates{({5*cos(deg(\a))},{5*sin(deg(\a))})};  %% closed hole          
          \addplot[color=penColor,fill=penColor,only marks,mark=*] coordinates{({5*cos(deg(\a))},{3*sin(deg(\a))})};  %% closed hole          
        \end{axis}
\end{tikzpicture}
\\
\begin{tikzpicture}
	\begin{axis}[
            %xmin=-.1,xmax=1.1,ymin=-.1,ymax=1.1,
            width=2.5in,
            clip=false,
            domain=(0:2*pi),
            axis lines=center,
            %ticks=none,
            unit vector ratio*=1 1 1,
            xlabel=$x$, ylabel=$y$,
            every axis y label/.style={at=(current axis.above origin),anchor=south},
            every axis x label/.style={at=(current axis.right of origin),anchor=west},
          ]      
          \pgfmathsetmacro{\a}{3*pi/10}
          \addplot [very thick, penColor, smooth] ({3*cos(deg(x))},{3*sin(deg(x))});
          \addplot [very thick, penColor, smooth] ({5*cos(deg(x))},{5*sin(deg(x))});
          \addplot [very thick, penColor, smooth] ({5*cos(deg(x))},{3*sin(deg(x))});
          
          \addplot[] plot coordinates {(0,0) ({5*cos(deg(\a))},{5*sin(deg(\a))})}; %% line

          \addplot[color=penColor,fill=penColor,only marks,mark=*] coordinates{({3*cos(deg(\a))},{3*sin(deg(\a))})};  %% closed hole          
          \addplot[color=penColor,fill=penColor,only marks,mark=*] coordinates{({5*cos(deg(\a))},{5*sin(deg(\a))})};  %% closed hole          
          \addplot[color=penColor,fill=penColor,only marks,mark=*] coordinates{({5*cos(deg(\a))},{3*sin(deg(\a))})};  %% closed hole          
        \end{axis}
\end{tikzpicture}
&
\begin{tikzpicture}
	\begin{axis}[
            %xmin=-.1,xmax=1.1,ymin=-.1,ymax=1.1,
            width=2.5in,
            clip=false,
            domain=(0:2*pi),
            axis lines=center,
            %ticks=none,
            unit vector ratio*=1 1 1,
            xlabel=$x$, ylabel=$y$,
            every axis y label/.style={at=(current axis.above origin),anchor=south},
            every axis x label/.style={at=(current axis.right of origin),anchor=west},
          ]      
          \pgfmathsetmacro{\a}{4*pi/10}
          \addplot [very thick, penColor, smooth] ({3*cos(deg(x))},{3*sin(deg(x))});
          \addplot [very thick, penColor, smooth] ({5*cos(deg(x))},{5*sin(deg(x))});
          \addplot [very thick, penColor, smooth] ({5*cos(deg(x))},{3*sin(deg(x))});
          
          \addplot[] plot coordinates {(0,0) ({5*cos(deg(\a))},{5*sin(deg(\a))})}; %% line

          \addplot[color=penColor,fill=penColor,only marks,mark=*] coordinates{({3*cos(deg(\a))},{3*sin(deg(\a))})};  %% closed hole          
          \addplot[color=penColor,fill=penColor,only marks,mark=*] coordinates{({5*cos(deg(\a))},{5*sin(deg(\a))})};  %% closed hole          
          \addplot[color=penColor,fill=penColor,only marks,mark=*] coordinates{({5*cos(deg(\a))},{3*sin(deg(\a))})};  %% closed hole          
        \end{axis}
\end{tikzpicture}
&
\begin{tikzpicture}
	\begin{axis}[
            %xmin=-.1,xmax=1.1,ymin=-.1,ymax=1.1,
            width=2.5in,
            clip=false,
            domain=(0:2*pi),
            axis lines=center,
            %ticks=none,
            unit vector ratio*=1 1 1,
            xlabel=$x$, ylabel=$y$,
            every axis y label/.style={at=(current axis.above origin),anchor=south},
            every axis x label/.style={at=(current axis.right of origin),anchor=west},
          ]      
          \pgfmathsetmacro{\a}{5*pi/10}
          \addplot [very thick, penColor, smooth] ({3*cos(deg(x))},{3*sin(deg(x))});
          \addplot [very thick, penColor, smooth] ({5*cos(deg(x))},{5*sin(deg(x))});
          \addplot [very thick, penColor, smooth] ({5*cos(deg(x))},{3*sin(deg(x))});
          
          \addplot[] plot coordinates {(0,0) ({5*cos(deg(\a))},{5*sin(deg(\a))})}; %% line

          \addplot[color=penColor,fill=penColor,only marks,mark=*] coordinates{({3*cos(deg(\a))},{3*sin(deg(\a))})};  %% closed hole          
          \addplot[color=penColor,fill=penColor,only marks,mark=*] coordinates{({5*cos(deg(\a))},{5*sin(deg(\a))})};  %% closed hole          
          \addplot[color=penColor,fill=penColor,only marks,mark=*] coordinates{({5*cos(deg(\a))},{3*sin(deg(\a))})};  %% closed hole          
        \end{axis}
\end{tikzpicture}
\end{tabular}
\]
\begin{enumerate}
\item Can you guess parametric formulas for the circles and for the ellipse?
\item Do you notice anything about the dots in the pictures? Can you explain why this happens?
\item Can you give a compass and straightedge construction that will give
you as many points on a given ellipse as you desire? Give a detailed explanation. 
\end{enumerate}
\end{prob}
\begin{teachingnote}
The compass and straightedge construction is optional.  (Not the main point.)
\end{teachingnote}


\begin{prob}
Can you give a parametric formula for this cool spiral? 
\[
\begin{tikzpicture}
	\begin{axis}[
            %xmin=-25,xmax=25,ymin=-25,ymax=25,
            width=3in,
            clip=false,
            axis lines=center,
            %ticks=none,
            unit vector ratio*=1 1 1,
            xlabel=$x$, ylabel=$y$,
            every axis y label/.style={at=(current axis.above origin),anchor=south},
            every axis x label/.style={at=(current axis.right of origin),anchor=west},
          ]      
          \addplot [very thick, penColor, smooth,samples=100,domain=(0:8*pi)] ({x*cos(deg(x))},{x*sin(deg(x))});
        \end{axis}
\end{tikzpicture}
\]
\end{prob}

\begin{prob}
Remind me once more, do the formulas that produce these plots define
functions? Discuss. Clearly identify the domain and range
as part of your discussion.
\end{prob}

