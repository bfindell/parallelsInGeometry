\newpage
\section{Verifying Our Constructions}

\fixnote{Perhaps a different or new activity here?  Or maybe problems 1 and 2 are not necessary.}

When we do our compass and straightedge constructions, we should take
care to verify that they actually work as advertised. We'll walk you
through this process. To start, remember what a circle is:

\begin{definition} 
A \textbf{circle} is the set of points that are a fixed distance from
a given point.
\end{definition}

\begin{prob} Is the center of a circle part of the circle?
\end{prob}

\begin{prob} 
Construct an equilateral triangle.  Why does this construction work?
\end{prob}




Now recall the SSS Theorem:

\begin{theorem}[SSS] 
Specifying three sides uniquely determines a triangle.
\end{theorem}



\begin{prob} Now we'll analyze the construction for copying angles. 
\begin{enumerate}
\item Use a compass and straightedge construction to duplicate an
  angle. Explain how you are really just ``measuring'' the sides of
  some triangle.
\item In light of the SSS Theorem, can you explain why the
  construction used to duplicate an angle works?
\end{enumerate}
\end{prob}


\begin{prob} Now we'll analyze the construction for bisecting angles.
\begin{enumerate}
\item Use compass and straightedge construction to bisect an
  angle. Explain how you are really just constructing (two)
  isosceles triangles. Draw these isosceles triangles in your figure.
\item Find two more triangles on either side of your angle bisector where
  you may use the SSS Theorem to argue that they have equal side
  lengths and therefore equal angle measures.
\item Can you explain why the construction used to bisect angles
  works?
\end{enumerate}
\end{prob}


Recall the SAS Theorem:

\begin{theorem}[SAS] 
Specifying two sides and the angle between them uniquely determines a
triangle.
\end{theorem}


\begin{prob} Now we'll analyze the construction for bisecting segments.
\begin{enumerate} 
\item Use a compass and straightedge construction to bisect a
  segment. Explain how you are really just constructing two
  isosceles triangles.
\item Note that the bisector divides each of the above isosceles
  triangles in half. Find two triangles on either side of your
  bisector where you may use the SAS Theorem to argue that they have
  equal side lengths and angle measures.
\item Can you explain why the construction used to bisect segments
  works?
\end{enumerate}
\end{prob}




\begin{prob} 
Now we'll analyze the construction of a perpendicular line through a
point not on the line.
\begin{enumerate}
\item Use a compass and straightedge construction to construct a
  perpendicular through a point. Explain how you are really just
  constructing an isosceles triangle.
\item Find two triangles in your construction where you may use the
  SAS Theorem to argue that they have equal side lengths and angle
  measures.
\item Can you explain why the construction used to construct a
  perpendicular through a point works?
\end{enumerate}
\end{prob}

