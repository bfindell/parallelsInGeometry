\newpage
\section{Constructible Numbers}
Compass and straightedge constructions involve drawing and finding intersections of two fundamental geometric objects:  lines and circles.  All more complicated constructions are combinations of pieces of these.  

In this activity, we explore what numbers are constructible (as lengths or distances) with compass and straightedge, assuming only that we begin with a segment of length 1.  We call such numbers \textit{constructible numbers}.  First we must establish how to do arithmetic with compass and straightedge.  

\subsection*{Arithmetic with Constructions}
\fixnote{For the multiplication one, we need a support picture to get them started.}
\begin{prob}
Suppose you are given a compass and a straightedge and segments of lengths $a$, $b$, and $1$.  
\begin{enumerate}
\item How would you construct a segment of length $a+b$? 
\item How would you construct a segment of length $a-b$? 
\item How would you construct a segment of length $ab$?  (Hint:  Use similar triangles.)  
\item How would you construct a segment of length $a\div b$? 
\item How would you construct a segment of length $\sqrt{a}$?  (Hint: Recall how to construct a geometric mean.)  
\end{enumerate}
\end{prob}

\begin{prob}
\fixnote{Here we a looking for a description (to your partner) of the construction process.}
Beginning with a segment of length 1, describe briefly how you might construct segments of the following lengths:
\begin{enumerate}
\item $\frac{7}{5}$
\item Any rational number, $p/q$
\item $3+2\sqrt{5}$
\item $\frac{3 + \sqrt{2-\sqrt{3}}}{1+\sqrt{5}}$
\end{enumerate}
\end{prob}
\begin{teachingnote}
This problem is about Seeing Structure in Expressions.
\end{teachingnote}

\begin{prob}
Based on the previous problems, if you begin with a segment of length 1, describe the set of all numbers constructible with the methods used so far.   
\end{prob}
\vspace{.5in}

\subsection*{Coordinate Constructions}
With the methods so far, we can construct neither $\sqrt[3]{2}$ nor $\pi$.  The question now is whether we have described the entire set of constructible numbers or whether there are additional constructions that will broaden our arithmetic and thereby enlarge the set.  

For this question, we turn to coordinate constructions, which allow us to use the methods of algebra to solve geometric problems.  A key habit here will be \textbf{imagining the algebra without actually doing it}---based on your extensive algebra experience with these kinds of problems.  

\fixnote{Need some teaching notes here.  Maybe the problems could work with fewer parts.}
\begin{prob}
Suppose you are given points $(p, q)$, and $(r, s)$ with integer coordinates.  
\begin{enumerate}
\item What arithmetic operations are involved in finding an equation $ax+by=c$ of the line containing these points?  
\item What can you conclude about the numbers $a$, $b$, and $c$? 
\item What if you begin with points that have coordinates that are rational numbers?  
\end{enumerate}
\end{prob}

\begin{prob}
Suppose you are given equations of the form 
$$ax+by = c$$
$$dx+ey=f$$
where $a$, $b$, $c$, $d$, $e$, and $f$ are all integers.  
\begin{enumerate}
\item What kind of geometric objects do these equations describe in the $xy$-plane?  
\item What arithmetic operations would you use to solve the equations simultaneously? 
\item What can you conclude about the numbers $x$ and $y$ that are the (simultaneous) solutions of these equations?  
\item How will your answers change if $a$, $b$, $c$, $d$, $e$, and $f$ are all rational numbers?  
\end{enumerate}
\end{prob}

\begin{prob}
Suppose you are given points $(h, k)$, and $(p, q)$ with integer coordinates?  
\begin{enumerate}
\item Write an equation of the circle with center $(h, k)$ and containing the point $(p, q)$?  
\item What arithmetic operations were involved in writing your equation of the circle?  
\item What can you conclude about the numbers that are coefficients in your equation?   
\end{enumerate}
\end{prob}

\begin{teachingnote}
Finding the intersection of a circle and a line involves a substitution and then solving a quadratic equation:  rational operations and extracting square roots.  Our students might not have enough experience with this.  

Finding the intersection of two circles involves first subtracting the two circle equations to remove all squared terms.  Then we have a  line.  Again, our students may have insufficient experience.  
\end{teachingnote}

\fixnote{Replace with specific examples that students actually do, step by step.}
\begin{prob}
Suppose you are given equations of the form 
$$x^2 + ax +y^2+by = c$$
$$x^2 + dx +y^2+ey = f$$
where $a$, $b$, $c$, $d$, $e$, and $f$ are all integers.  
\begin{enumerate}
\item What kind of geometric objects do these equations describe in the $xy$-plane?  
\item What arithmetic operations would you use to solve the equations simultaneously? 
\item What can you conclude about the numbers $x$ and $y$ that are the (simultaneous) solutions of these equations?  
\item How will your answers change if $a$, $b$, $c$, $d$, $e$, and $f$  are all rational numbers?  
\end{enumerate}
\end{prob}

\fixnote{The next two questions frame the whole activity.  Perhaps place these up front in some way.  The gist:  We know what objects we have and what tools we have.  What number can come out of it.  After drawing a circle with circumference pi, we can't unwrap it because we don't have string as one of our tools.}
\begin{prob}
Based on the previous problems, if you begin with a coordinate system with only integer coordinates, how would you describe the set of all numbers (coordinates) that are constructible via lines and circles?  
\end{prob}

\begin{prob}
Considering that all compass and straightedge constructions are about lines, circles, and their intersections, what do your results about coordinate constructions imply about compass and straightedge constructions?  
\end{prob}

\begin{prob}
Name some numbers that are \textbf{not constructible} with compass and straightedge.  
\end{prob}
