\newpage
\section{Constructible Numbers}
Compass and straightedge constructions involve drawing and finding intersections of two fundamental geometric objects:  lines and circles.  All more complicated constructions are combinations of pieces of these.  

In this activity, we explore what numbers are constructible (as lengths or distances) with compass and straightedge, assuming only that we begin with a segment of length 1.  We call such numbers \textit{constructible numbers}.  First we must establish how to do arithmetic with compass and straightedge.  
\fixnote{When do we think of $a$ and $b$ first as whole number, then integers, then rationals, then surds.  Condense the whole thing somewhat.  Provide structure that highlights the argument.  Need teaching note about intersecting circles and lines.}  
\begin{prob}
Suppose you are given a compass and a straightedge and segments of lengths $a$, $b$, and $1$.  
\begin{enumerate}
\item How would you construct a segment of length $a+b$? 
\item How would you construct a segment of length $a-b$? 
\item How would you construct a segment of length $ab$?  (Hint:  Use similar triangles.)  
\item How would you construct a segment of length $a\div b$? 
\item How would you construct a segment of length $\sqrt{a}$?  (Hint: Recall how to construct a geometric mean.)  
\end{enumerate}
\end{prob}

\begin{prob}
Beginning with a segment of length 1, describe briefly how you might construct segments of the following lengths.
\begin{enumerate}
\item $\frac{7}{5}$
\item $3+2\sqrt{5}$
\item $\frac{4+\sqrt{5}}{3}$
\item $\frac{3 + \sqrt{2-\sqrt{3}}}{1+\sqrt{5}}$
\end{enumerate}
\end{prob}

\teachingnote{This problem is about Seeing Structure in Expressions.}

\begin{prob}
Based on the previous problems, if you begin with a segment of length 1, how would you describe the set of all numbers constructible with methods used so far?    
\end{prob}

Note that with the methods so far, we can construct neither $\sqrt[3]{2}$ nor $\pi$.  The question now is whether we have described the whole set of constructible numbers or whether there are additional constructions that will broaden our arithmetic and thereby enlarge the set.  

For this question, we turn to coordinate constructions, which allow us to use the methods of algebra to solve geometric problems.  A key habit here will be \emph{imagining the algebra without actually doing it}---based on your extensive algebra experience with these kinds of problems.  

\begin{prob}
Suppose you are given points $(p, q)$, and $(r, s)$ with integer coordinates.  
\begin{enumerate}
\item What arithmetic operations are involved in finding an equation $ax+by=c$ of the line containing these points?  
\item What can you conclude about the numbers $a$, $b$, and $c$? 
\item What if you begin with points that have coordinates that are rational numbers?  
\end{enumerate}
\end{prob}

\begin{prob}
Suppose you are given equations of the form 
$$ax+by = c$$
$$dx+ey=f$$
where $a$, $b$, $c$, $d$, $e$, and $f$ are all integers.  
\begin{enumerate}
\item What kind of geometric objects do these equations describe in the $xy$-plane?  
\item What arithmetic operations would you use to solve the equations simultaneously? 
\item What can you conclude about the numbers $x$ and $y$ that are the (simultaneous) solutions of these equations?  
\item How will your answers change if $a$, $b$, $c$, $d$, $e$, and $f$ are all rational numbers?  
\end{enumerate}
\end{prob}

\begin{prob}
Suppose you are given points $(h, k)$, and $(p, q)$ with integer coordinates?  
\begin{enumerate}
\item Write an equation of the circle with center $(h, k)$ and containing the point $(p, q)$?  
\item What arithmetic operations were involved in writing your equation of the circle?  
\item What can you conclude about the numbers that are coefficients in your equation?   
\end{enumerate}
\end{prob}

\begin{prob}
Suppose you are given equations of the form 
$$x^2 + ax +y^2+by = c$$
$$x^2 + dx +y^2+ey = f$$
where $a$, $b$, $c$, $d$, $e$, and $f$ are all integers.  
\begin{enumerate}
\item What kind of geometric objects do these equations describe in the $xy$-plane?  
\item What arithmetic operations would you use to solve the equations simultaneously? 
\item What can you conclude about the numbers $x$ and $y$ that are the (simultaneous) solutions of these equations?  
\item How will your answers change if $a$, $b$, $c$, $d$, $e$, and $f$  are all rational numbers?  
\end{enumerate}
\end{prob}

\begin{prob}
Based on the previous problems, if you begin with a coordinate system with only integer coordinates, how would you describe the set of all numbers (coordinates) that are constructible via lines and circles?  
\end{prob}

\begin{prob}
Considering that all compass and straightedge constructions are about lines, circles, and their intersections, what do your results about coordinate constructions imply about compass and straightedge constructions?  
\end{prob}

\begin{prob}
Name some numbers that are \textbf{not constructible} with compass and straightedge.  
\end{prob}

The idea that some numbers are not constructible is exactly what was needed to address several problems first posed by the Greeks in antiquity, such as doubling the cube and trisecting an angle.  In a paper published in 1987, Pierre Wantzel used algebraic methods to prove the impossibility of these geometric constructions.  

\begin{prob}
Suppose you have a square of side length $s$ and you want to ``double the square.''  In other words, you want to construct a square with \textbf{twice the area}.  
\begin{enumerate}
\item What is the side length of the desired square?  Explain your reasoning. 
\item Is this side length constructible?  Explain.  
\end{enumerate}
\end{prob}

\begin{prob}
Suppose you have a cube of side length $s$ and you want to ``double the cube.''  In other words, you want to construct a cube with \textbf{twice the volume}.  
\begin{enumerate}
\item What is the side length of the desired cube?  Explain your reasoning. 
\item Is this side length constructible?  Explain.  
\end{enumerate}
\end{prob}

\begin{prob}
You may remember some double angle formulas from trigonometry.  There are also triple angle formulas.  For example, for any angle $\theta$,  $\cos3\theta=4\cos^3\theta -3\cos\theta$.  
\begin{enumerate}
\item What is $\cos60^\circ$?
\item Write the above triple angle formula for $\theta = 20^\circ$.  %% solution of the equation $4x^3-3x=1/2$, 
\item Explain why $x = \cos20^\circ$ must be a root of the polynomial $8x^3-6x-1$.  
\item Explain how the rational root theorem implies that this polynomial has no linear factors.  
\item Explain why this polynomial must therefore be irreducible over the rational numbers.  
\item You may recall from Math 1165 that some methods of solving cubic equations involve extracting cube roots.  What does this imply about trisecting angles?  
\item You may recall, from earlier this semester, discussing a method for trisecting an angle with paper folding.  What does that method imply about the relationship between the numbers that are constructible by paper folding and those that are constructible by compass and straightedge?  Explain.  
\end{enumerate}
\end{prob}
