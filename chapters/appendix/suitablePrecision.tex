\newpage
\section{Suitable Precision in Language}

\fixnote{Maybe modify the activity for "Two different segments."  (What did I mean by that?) 
Also:  An angle is is not just the segments.   Need rays and interior.  Given angles defined by segments of various lengths, ask which is the bigger angle.}

\begin{teachingnote}
This activity is intended to remind students that degrees measure angles and that an angle requires a vertex. 
\end{teachingnote}

\begin{prob}
Explain the geometric distinction between a segment and its length.  How are the two usually denoted differently?  
\end{prob}
\vspace{.4in}

\begin{prob}
There are (at least) two ways of thinking about angles.  
\begin{enumerate}
\item Use precise language to describe an angle as a set of points.  
\vspace{.4in}
\item Use precise language to describe an angle as an amount of turning.  
\end{enumerate}
\vspace{.4in}
\end{prob}
\begin{teachingnote}
An angle is the union of two rays with a common endpoint, which is called the vertex of the angle.  The vertex of an angle can also be considered the center of a rotation that would map one ray that the other.  
\end{teachingnote}

\begin{prob}
Explain the geometric distinction between an angle and its measure.  How are the two usually denoted differently?  And how do your answers relate to the previous problem?  
\end{prob}
\vspace{.8in}

\newpage
\begin{prob}
Use your meanings for angles to improve upon the following imprecise statements. 

{\renewcommand{\arraystretch}{1.5}
\begin{tabular}{|>{\centering\arraybackslash}m{4cm}|>{\centering\arraybackslash}m{9.5cm}|>{\centering\arraybackslash}m{4cm}|}\hline
Statement & Improved Version & Comments \\\hline

\rule{0pt}{1cm}A triangle has $180^\circ$. & & \\ \hline

\rule{0pt}{1cm}A line measures $180^\circ$. & & \\ \hline

\rule{0pt}{1cm}A circle is (or has) $360^\circ$. & & \\ \hline
 \hline
\end{tabular}}
\end{prob}

\begin{teachingnote}
\begin{itemize}
\itemsep0em
\item Let students struggle to figure out what is imprecise about the statements in the problem.  
\item The three vertices of the triangle are vertices of the three interior angles to be measured (and then summed).  
\item On a line, any point may be considered the vertex of a straight angle.
\item For a circle, we need its center, which is the vertex of central angles 
that can sum to $360^\circ$.  
\end{itemize}
\end{teachingnote}






