\newpage
\section{Suitable Precision in Language}

\fixnote{Make the last problem (about angles) first and use it to ground the other problems.}  
\begin{teachingnote}
Some students need to be reminded that degrees measure angles and that an angle requires a vertex. 
\begin{itemize}
\itemsep0em
\item Let students struggle to figure out what is imprecise about the statements in the problem.  
\item The three vertices of the triangle are vertices of the three interior angles to be measured (and then summed).  
\item On a line, identify any point to be the vertex of a straight angle.  
\item For a circle, we need its center, which is the vertex of central angles 
that can sum to $360^\circ$.  
\end{itemize}
\end{teachingnote}

\begin{prob}
Improve upon the following imprecise statements. 

{\renewcommand{\arraystretch}{1.5}
\begin{tabular}{|>{\centering\arraybackslash}m{4cm}|>{\centering\arraybackslash}m{9.5cm}|>{\centering\arraybackslash}m{4cm}|}\hline
Statement & Improved Version & Comments \\\hline

\rule{0pt}{1cm}A triangle has $180^\circ$. & & \\ \hline

\rule{0pt}{1cm}A line measures $180^\circ$. & & \\ \hline

\rule{0pt}{1cm}A circle is (or has) $360^\circ$. & & \\ \hline
 \hline
\end{tabular}}
\end{prob}

\begin{prob}
Explain the geometric distinction between a segment and its length.  How are the two usually denoted differently?  
\end{prob}
\vspace{.8in}

\begin{prob}
Explain the geometric distinction between an angle and its measure.  How are the two usually denoted differently?  
\end{prob}
\vspace{.8in}
\begin{prob}
There are (at least) two ways of thinking about angles:  (1) as a set of points, and (2) as an amount of turning.  
Describe how you have used both of these ways of thinking in this activity.  
\end{prob}

