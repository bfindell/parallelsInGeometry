\newpage
\section{Suitable Precision in Language (aka A.3B)}

\fixnote{Need a better intro or scaffolding so that they know that degrees means angle and angle require a vertex.}  

\begin{prob}
Improve upon the following imprecise statements. 

{\renewcommand{\arraystretch}{1.5}
\begin{tabular}{|>{\centering\arraybackslash}m{4cm}|>{\centering\arraybackslash}m{9.5cm}|>{\centering\arraybackslash}m{4cm}|}\hline
Statement & Improved Version & Comments \\\hline

\rule{0pt}{1cm}A triangle has $180^\circ$. & & \\ \hline

\rule{0pt}{1cm}A line measures $180^\circ$. & & \\ \hline

\rule{0pt}{1cm}A circle is (or has) $360^\circ$. & & \\ \hline
 \hline
\end{tabular}}
\end{prob}

\begin{prob}
Explain the geometric distinction between a segment and its length.  How are the two usually denoted differently?  
\end{prob}
\vspace{.8in}

\begin{prob}
Explain the geometric distinction between an angle and its measure.  How are the two usually denoted differently?  
\end{prob}
\vspace{.8in}
\begin{prob}
There are (at least) two ways of thinking about angles:  (1) as a set of points, and (2) as an amount of turning.  
Describe how you have used both of these ways of thinking in this activity.  
\end{prob}

