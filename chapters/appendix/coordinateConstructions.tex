\newpage
\section{Coordinate Constructions}
In synthetic geometry, point, line and plane are taken to be undefined terms.  In analytic (coordinate) geometry, in contrast, we make the following definitions.  
\begin{definition}
A \emph{point} is an ordered pair $(x,y)$ of real numbers. A \emph{line} is the set of ordered pairs $(x,y)$ that satisfy an equation of the form $ax + by = c$, where $a$, $b$, and $c$ are real numbers and $a$ and $b$ are not both 0.   
\end{definition}

Many of the problems below are expressed generally.  You may find it useful to try some specific examples before the general case.  

\begin{prob}
In the above definition of a line in coordinate geometry, why is it important to require that $a$ and $b$ are not both 0?  
\end{prob}

\begin{prob}
Given points $(x_1, y_1)$ and $(x_2, y_2)$, find the distance between them in the coordinate plane.
\end{prob}

\begin{prob}
Find the midpoint of the segment from $(x_1, y_1)$ and $(x_2, y_2)$.  Explain why your formula makes sense. 
\end{prob}

\begin{teachingnote}
Probably out of habit from the slope formula, some students will subtract coordinates to find midpoints.  This is a good place to connect algebraically various ways of finding the midpoint of two values, such as (1) taking half the difference and adding it to the lower number; (2) adding the two numbers and dividing by two. 
\end{teachingnote}   

\begin{prob}
Recall that in synthetic geometry, a circle is defined as the set of points that are equidistant from a center.  Use this definition to determine the equation of circle with center $(h, k)$ and radius $r$.\standardhs{G-GPE.1}  
\end{prob}

\begin{prob}
For each pair of points below, find an equation of the line containing the two points.  
\begin{enumerate}
\item Points $(2,3)$ and $(5,7)$.  
\item Points $(2,3)$ and $(2,7)$.  
\item Points $(2,3)$ and $(5,3)$. 
\item Points $(x_1, y_1)$ and $(x_2, y_2)$.  
\end{enumerate}
\end{prob}

\begin{prob}
Express each of your previous equations in the form $ax + by = c$ and also in the form $y = mx + b$.   What are the advantages and disadvantages of these forms?  
\end{prob}

\begin{prob}
In school mathematics, lines are usually of the form $y = mx + b$.  Why is it unambiguous to talk about \emph{the slope} of such a line?  In other words, given a non-vertical line in the plane, explain why any two points on the line will yield the same slope.\standard{8.EE.6}  
\end{prob}

