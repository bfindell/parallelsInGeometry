\newpage

\section{Triangle Investigation}

\begin{teachingnote}
Preactivity:  Do the first three of these at home.  The upshot is triangle congruence:  three measures are (often) enough.  

Some students will need to be reminded that, e.g., $B$ is vertex of $\angle ABC$.
\end{teachingnote}

\begin{prob}
Draw triangles satisfying the conditions given below.  You may use whatever tools you like (e.g., ruler, protractor, compass, sticks, tracing paper, or Geogebra).  

In each part, use reasoning to determine whether the information provided determines a unique $\triangle ABC$, more than one triangle, or no triangle.\standard{7.G.2}   Note:  To check to see if two triangles are the same, attempt to lay one directly on top of the other.  

\begin{enumerate}

\item $AB = 4$ and $BC = 5$
\item $m\angle CAB = 25^\circ$, $m\angle ABC = 75^\circ$, $m\angle BCA = 80^\circ$
\item $m\angle CAB = 25^\circ$, $m\angle ABC = 65^\circ$, $m\angle BCA = 80^\circ$
\item $AB = 5$, $m\angle BAC = 30^\circ$, $m\angle ABC = 45^\circ$
\item $AB = 5$, $BC = 4$, $m\angle ABC = 60^\circ$
\item $BC = 7$, $CA = 8$, $AB = 9$
\item $BC = 4$, $CA = 8$, $AB = 3$
\item $m\angle ABC = 45^\circ$, $BC = 8$, $CA = 12$
\item $m\angle ABC = 30^\circ$, $BC = 10$, $CA = 7$
\item $m\angle ABC = 60^\circ$, $BC = 10$, $CA = 3$

\end{enumerate}

\end{prob}
