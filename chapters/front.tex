

\newpage

\begin{fullwidth}
~\vfill
\thispagestyle{empty}
\setlength{\parindent}{0pt}
\setlength{\parskip}{\baselineskip}
Copyright \copyright~2022, Bart Snapp and Bradford Findell

\vspace{.5cm}

\noindent
This work is licensed under the Creative Commons:
\begin{center}
Attribution-NonCommercial-ShareAlike License 
\end{center}
To view a copy of this license, visit
\[
\texttt{http://creativecommons.org/licenses/by-nc-sa/3.0/}
\]


\vspace{.5cm}
\noindent This document was typeset on \today.
\end{fullwidth}


\chapter*{Preface}
\addcontentsline{toc}{chapter}{Preface}


These notes are designed with future middle grades mathematics
teachers in mind.  While most of the material in these notes would be
accessible to an accelerated middle grades student, it is our hope
that the reader will find these notes both interesting and
challenging.  In some sense we are simply taking the topics from a
middle grades class and pushing ``slightly beyond'' what one might
typically see in schools. In particular, there is an emphasis on the
ability to communicate mathematical ideas.  Three goals of these notes
are:
\begin{itemize}
\item To enrich the reader's understanding of both numbers and algebra. 
From the basic algorithms of arithmetic---all of which have algebraic
underpinnings---to the existence of irrational numbers, we hope to show
the reader that numbers and algebra are deeply connected.
\item To place an emphasis on problem solving. The reader will be exposed 
to problems that ``fight-back.'' Worthy minds such as yours deserve
worthy opponents. Too often mathematics problems fall after a single
``trick.'' Some worthwhile problems take time to solve and cannot be done
in a single sitting.
\item To challenge the common view that mathematics is a body of knowledge 
to be memorized and repeated. The art and science of doing mathematics
is a process of reasoning and personal discovery followed by
justification and explanation. We wish to convey this to the reader,
and sincerely hope that the reader will pass this on to others as
well.
\end{itemize}
In summary---you, the reader, must become a doer of mathematics.  To
this end, many questions are asked in the text that follows. Sometimes
these questions are answered; other times the questions are left for
the reader to ponder. To let the reader know which questions are left
for cogitation, a large question mark is displayed:
\QM
The instructor of the course will address some of these questions. If
a question is not discussed to the reader's satisfaction, then we
encourage the reader to put on a thinking-cap and think, think, think!
If the question is still unresolved, go to the World Wide Web and
search, search, search!

This document is open-source. It is licensed under the Creative
Commons Attribution-NonCommercial-ShareAlike (CC BY-NC-SA)
License. Loosely speaking, this means that this document is available
for free. Anyone can get a free copy of this document 
from the following sites:
\[
\texttt{http://www.math.osu.edu/\~{}snapp/1165/}
\]
\[
\texttt{http://www.math.osu.edu/\~{}findell.2}
\]
Please report corrections, suggestions, gripes, complaints, and
criticisms to Bart Snapp at \texttt{snapp@math.osu.edu} or Brad Findell
at \texttt{findell.2@osu.edu}


\section*{Thanks and Acknowledgments}

This document has a somewhat lengthy history. In the Fall of 2005 and
Spring of 2006, Bart Snapp gave a set of lectures at the University of
Illinois at Urbana-Champaign. His lecture notes were typed and made
available as an open-source textbook. During subsequent semesters, those notes
were revised and modified under the supervision of Alison Ahlgren and
Bart Snapp. Many people have made contributions,
including Tom Cooney, Melissa Dennison, and Jesse Miller. A number of
students also contributed to that document by either typing original
hand-written notes or suggesting problems. They are: Camille Brooks,
Michelle Bruno, Marissa Colatosti, Katie Colby, Anthony `Tino'
Forneris, Amanda Genovise, Melissa Peterson, Nicole Petschenko, Jason
Reczek, Christina Reincke, David Seo, Adam Shalzi, Allice Son, Katie
Strle, and Beth Vaughn.

In 2009, Greg Williams, a Master of Arts in Teaching student at
Coastal Carolina University, worked with Bart Snapp to produce an
early draft of the chapter on isometries.

In the Winter of 2010 and 2011, Bart Snapp gave a new set of lectures
at the Ohio State University. In this course the previous lecture
notes were heavily modified, resulting in a new text \textit{Parallels
in Geometry}.  Since 2012, Bart Snapp and Brad Findell have
continued revising these notes annually.  In particular, during 2014 and 2015, 
exposition and activities were added to address ideas from the Common 
Core State Standards (CCSS).  Many of the individual standards are 
included as margin notes that begin ``CCSS.''  

\makeatletter %% adds space so that the numbers of the toc don't bump
\renewcommand{\l@section}{\@dottedtocline{1}{5em}{5em}}
\renewcommand{\l@subsection}{\@dottedtocline{2}{5em}{5em}}
\renewcommand{\l@subsubsection}{\@dottedtocline{3}{5em}{5em}}
\makeatother


\setcounter{tocdepth}{1}
\tableofcontents
