\newpage

\section{Math 4407 Exam Review}
The exam will sample from Chapters 1 -- 4 and Activities 1--32 and 38 of your course notes.  Exam problems will focus on Euclidean geometry.  In a few of the problems, you will be asked \emph{to compare the Euclidean result to spherical or hyperbolic geometry and to explain your reasoning.}  
\subsection{Review Ideas}

\begin{itemize}\itemsep0em
\item Be able to state definitions of commonly used terms, such as 
\begin{itemize}
\item Perpendicular line, parallel line, line segment, ray, angle, circle
\item Concurrent, collinear
\item Equilateral, equiangular, and regular polygons
\item Acute, obtuse, and right angles and triangles; isosceles and scalene triangles
\item Straight, complementary, and supplementary angles
\item Trapezoid (inclusive and exclusive), parallelogram, rhombus, rectangle, square, kite
\item Median, perpendicular bisector, angle bisector, altitude
\item Centroid, circumcenter, incenter, orthocenter
\item Chord, arc, arc measure, central angle, inscribed angle, tangent
\end{itemize}
\item Be able to perform standard constructions and explain why they work. 
\item Be able to draw (careful) figures satisfying particular conditions.  
\item Be able to explain key concepts such as area and angle.   
\item Be able to state precisely the triangle congruence criteria. 
\item Know the properties of various special quadrilaterals and be able to prove them.  
\item Know the various centers of a triangle, how to construct them, and whether they can lie outside the triangle.  
\item Be able to state key theorems and prove them in at least two (2) ways, especially:  
\begin{itemize}
\item Isosceles triangle theorem and its converse
\item Pythagorean theorem and its converse
\item The angle sum of a triangle 
\end{itemize}

\item Know the distinction between synthetic and analytic geometry.
\item Know the basic rigid motions, what is required to specify them, and their properties. 
\item Know what it means to say that transformations of the plane are functions.  
\item Know how to define congruence in terms of basic rigid motions. 
\item Know how to define similarity in terms of dilations and basic rigid motions.  
\item Know and be able to use criteria for congruence and similarity of triangles.  
\item Know how to use transformations  to describe symmetries of figures, including tessellations.  
%\item Know the focus and directrix definition of parabola and be able to use it in both synthetic and analytic geometry.
\item Know the definition of circle and be able to use it in both synthetic and analytic geometry.
\item Be aware of assumptions underlying Euclidean geometry and how those assumptions can be different in other geometries (such as spherical geometry).  
\item Use similarity to find missing lengths by reasoning from the scale factor or from within-figure comparisons.   
\item Understand right-triangle trigonometry as similarity, and use trigonometry to solve problems.  (See activity A.27 for a review.)
%\item Reason about length, area, and volume in similarity situations.  How are rep-tiles related to this question?  
%\item Use shearing and Cavalieri's principle to reason about area and volume.  
%\item If you know the area of a rectangle, what can you say about its perimeter?  What about more general figures?  
%\item If you know the perimeter of a rectangle, what can you say about its area?  What about more general figures? 

\end{itemize}

\subsection*{Review Problems}
\begin{prob}
Describe Euclid's (compass and straightedge) construction for an equilateral triangle, and explain
why it works.
\end{prob}

%\begin{prob}
%Use the picture below to show that a pair of medians intersects at a point 2/3 of the way from the vertex to the opposite side.  Then use that fact to argue that the three medians must be concurrent.  
%$$\includegraphics[width=2.5in]{../graphics/median1.pdf}$$
%\end{prob}

\begin{prob}
Construct a $30$-$60$-$90$ right triangle. Explain the steps in your
  construction and how you know it works.
\end{prob}

\begin{prob}
Construct a $45$-$45$-$90$ right triangle. Explain the steps in your
  construction and how you know it works.
\end{prob}

\begin{prob}
Prove that the points on an angle bisector are \emph{exactly those} that are equidistant from the sides of the angle. 
\end{prob}

\begin{prob} 
Concurrency of angle bisectors. 
\begin{enumerate}
\item For an arbitrary triangle, draw carefully to demonstrated that the angle bisectors of a triangle are concurrent at the incenter.   
\item Prove that the angle bisectors of a triangle are concurrent.  (Hint:  You may use the result, proved in lecture, that the points on an angle bisector are exactly those that are equidistant from the sides of the angles.)  
\end{enumerate}
\end{prob} 


\begin{prob}
Where is the orthocenter of a right triangle?  Explain your reasoning.  What about the circumcenter?  Again, explain your reasoning. 
\end{prob}

\begin{prob}
Show that, given any three non-collinear points in the Euclidean plane, there is a unique circle passing through the three points.
\end{prob}

\begin{prob}
Prove: A radius that is perpendicular to a chord bisects the chord. 
\end{prob}

\begin{prob}
Prove:  A radius that bisects a chord is perpendicular to the chord. 
\end{prob}

\begin{prob}
Given a circle, give a construction that finds its center.
\end{prob}

\begin{prob}
State and prove a condition about the opposite angles of any quadrilateral that is inscribed in a circle.  
\end{prob}

\begin{prob}
Construct a tangent line from a point outside a given circle to the circle.
\end{prob}

\begin{prob}
Give an informal derivation of the relationship between the circumference and area of a circle. 
\end{prob}

\begin{prob}
Prove:  If a quadrilateral is a parallelogram, then opposite sides are congruent.
\end{prob}

\begin{prob}
Prove:  If opposite sides of a quadrilateral are congruent, then it is a parallelogram.
\end{prob}

\begin{prob}
Claim:  The diagonals of a rhombus are perpendicular. 
\begin{enumerate}
\item Prove the claim. 
\item State the converse of the claim. 
\item Is the converse true?  If so, prove it.  If not, ``salvage it'' to make a true statement, and prove it.  
\end{enumerate}
\end{prob}


\begin{prob}
Draw an arbitrary convex quadrilateral.  Form a second quadrilateral by connecting the midpoints of the sides 
of the first quadrilateral.  You will notice that the second quadrilateral is a special quadrilateral. Make a conjecture about the second quadrilateral and prove it.  
\end{prob}

\begin{prob}
The following picture shows a triangle that has been folded
  along the dotted lines:
\[
\includegraphics{../graphics/origamiPBPTri.pdf}
\]
Explain how the picture ``proves'' the following statements:
\begin{enumerate}
\item The interior angles of a triangle sum to $180^\circ$. 
\item The area of a triangle is given by $bh/2$. 
\end{enumerate}
\end{prob}

%\begin{prob}
%The figure below illustrates a construction given an angle $\theta$ and a segment $\overline{AB}$.  Line $\overleftrightarrow{DE}$ is the perpendicular bisector of $\overline{AB}$, and $\overleftrightarrow{BE}$ is perpendicular to $\overrightarrow{BC}$, as marked.  
%\[
%\includegraphics[scale=0.5]{../graphics/trickyConstruction}
%\]
%\begin{enumerate}
%%\item Mark the diagram and state what must be true about the construction so that $\angle\alpha=\angle\beta$. 
%\item Prove that $\angle\theta=\angle\beta$. 
%\item What will happen to $\angle\beta$ if point $X$ is moved over to point $Y$?
%\end{enumerate}
%\end{prob}

%
%\begin{prob}
%During a solar eclipse we see that the apparent diameter of the Sun and Moon are nearly equal. If the Moon is around 240,000 miles from Earth, the Moon's diameter is about 2000 miles, and the Sun's diameter is about 865,000 miles how far is the Sun from the Earth?
%\begin{enumerate}
%\item Draw a relevant (and helpful) picture showing the important points of this problem.
%\item Write an expression that gives the solution to this problem---show all work.
%\end{enumerate}
%\end{prob}
%
%\begin{prob}
%David proudly owns a 42 inch (measured diagonally) flat screen
%  TV. Michael proudly owns a 13 inch (measured diagonally) flat screen
%  TV. Dave sits comfortably with his dog Fritz at a distance of 10
%  feet. How close must Michael sit from his TV to have the ``same''
%  viewing experience?  Explain your reasoning.
%\begin{enumerate}
%\item Draw a relevant (and helpful) picture showing the important
%  points of this problem.
%\item Solve this problem, and be sure to explain your reasoning.
%\end{enumerate}
%\end{prob}

%\begin{prob}
%A typical adult male gorilla is about 5.5 feet tall and weighs about 400 pounds. Suppose King Kong was about 22 feet tall and proportioned like a typical adult male gorilla.
%\begin{enumerate}
%\item Write an expression that approximates King Kong's weight. Briefly explain your reasoning.
%\item The circumference of the neck of a typical adult male gorilla is 36 inches. Approximately what would be the circumference of King Kong's neck? Briefly explain your reasoning.
%\item Suppose an Ohio State sweatshirt for a typical adult male gorilla requires 3 square yards of fabric.  Approximately how much fabric would be required for an Ohio State sweatshirt for King Kong?  Briefly explain your reasoning.
%\end{enumerate}
%\end{prob}
%
%\begin{prob}
%Brenah is drinking fruit punch from a glass shaped like an inverted cone.  Suppose the glass has a height 5 in. and a base of radius 2 in.  What is the volume of the glass?  What is the height of the fruit punch when the glass is half full?  Generalize your result for any glass shaped like an inverted cone.  
%\end{prob}

%\begin{prob}
%A cup has a circular opening, a circular base, and circular cross sections at every height parallel to the base.  The opening has a diameter of 9 cm, the base has a diameter of 6 cm, and the cup is 12 cm high.  
%What is the volume of the cup?  Explain your reasoning.  
%If the cup is filled to half its height, what fraction of the cup's volume is filled?  Explain your reasoning.  
%\end{prob}

%\begin{prob}
%Suppose you use a photocopier to enlarge a figure to 125\% of its original size.  What is the scale factor of the enlargement?  What happens to areas under the enlargement?  If you lost the original figure, what reduction percentage would you use on the enlargement to create a figure congruent to the original?  What is the scale factor for the reduction?  
%\end{prob}

\begin{prob}
Explain how the following picture ``proves'' the Pythagorean Theorem.
\[
\includegraphics[scale=0.75]{../graphics/pbpdilation.pdf}
\]
\end{prob}

%\begin{prob}
%Here is a right triangle, note it is \textbf{not} drawn to scale:
%\[
%\includegraphics{../graphics/origamiSimQ.pdf}
%\]
%Solve for all unknowns in the following cases.
%\begin{enumerate}
%\item $a = 3$, $b = ?$, $c = ?$, $d = 12$, $e = 5$, $f = ?$
%\item $a = ?$, $b = 3$, $c = ?$, $d =8$, $e = 13$, $f = ?$
%\item $a = 7$, $b = 4$, $c = ?$, $d =?$, $e = 11$, $f = ?$
%\item $a = 5$, $b = 2$, $c = ?$, $d =6$, $e = ?$, $f = ?$
%\end{enumerate}
%In each case explain your reasoning.
%\end{prob}

%\begin{prob}
%Describe a general (and foolproof) way of demonstrating that any two parabolas are similar.
%\end{prob}
%
%\begin{prob}
%Construct a tangent line from a point outside a given circle to the circle.
%\end{prob}
%
%\begin{prob}
%Explain how the formula for the volume of a sphere follows from the formula for the volume of a cone and Cavalieri's Principle.
%\end{prob}
%

\begin{prob}
Given a figure and a rotation of that figure, find the center and angle of rotation.  
\end{prob}

%\begin{prob}
%In the figure below  $O$ is the center of the circle, $\overline{XY}$ is a diameter, $a = PX$, $b=PY$, and $c=PZ$.  
%$$\includegraphics[scale=0.6]{../graphics/means}$$
%\begin{enumerate}
%\item Show that $c=\sqrt{ab}$.  
%\item Use the figure to explain the Arithmetic-Geometric Mean Inequality: $\frac{a+b}{2} \ge \sqrt{ab}$.  
%%\item Why is this called the Arithmetic-Geometric Mean Inequality?  
%\end{enumerate}
%\end{prob}

%\begin{prob}
%If the perimeter of a rectangle is 20 feet, what is the most one can say about the rectangle's area?  If the perimeter of any simple closed 2-dimensional shape is 20 feet, what is the most anyone can say about its area?
%\end{prob}
%
%\begin{prob}
%If the surface area of a rectangular prism is 20 square feet, what is the most one can say about the prism's volume?  If the surface area of any simple closed 3-dimensional shape is 20 square feet, what is the most one can say about its volume? 
%\end{prob}

%\begin{prob}
%Why do cute furry animals curl up to stay warm in the winter?  Why are most ugly desert reptiles long and skinny?
%\end{prob}
%
%\begin{prob}Simple closed curve A is contained entirely inside simple closed curve B.  
%\begin{enumerate}
%\item True or False:  The area enclosed by A is less than the area enclosed by B. Explain
%\item True or False:  The perimeter of A is less than the perimeter of B. Explain.  
%\end{enumerate}
%\end{prob}

%\begin{prob}
%Is it correct to say that ``area is length times width''?  Think about what these three quantities mean.  When would it be correct in the numerical sense and why?  (Make sure you use the meaning of multiplication.)   
%\end{prob}

%\begin{prob}
%The apothem of a regular polygon is defined to be the shortest distance from the center of the polygon to an edge.
%\begin{enumerate}
%\item There is a nice relationship between the apothem, perimeter, and area for a regular polygon.  See if you can find it. (Hint:  Split the polygon into congruent triangles from its center and find the area of the polygon in terms of the apothem and perimeter.)  You can assume you know the area of a triangle $=\frac{1}{2}$(Length of Base)(Length of Height).
%\item What does this result say about the area of a circle?  Explain. (Assume you know the circumference of a circle is $2\pi(radius)$.)
%\end{enumerate}
%\end{prob}

%\begin{prob}
% Is it correct to say that ``volume is length times width times height''? What must be true about a figure so that the numerical volume can be more easily measured by ``area times height''?
%\end{prob}
%
%\begin{prob}
%Are there figures for which there is no formula for measuring length, area, and volume?  Explain.  What does your answer to this question imply about the teaching of geometric measurement?
%\end{prob}
%
%\begin{prob}
%Convert 25 yards to meters (and 25 meters to yards) using ``2.54 cm in each inch'' as the only Metric-English unit conversion.  Now convert 25 square yards to square meters and 25 square meters to square yards.  Do the same with cubic yards and cubic meters.
%\end{prob}

%\begin{prob}
%In track and field, 1600 meters is often called the ``one mile,'' but this is not exactly correct.  Is 1600 meters longer or shorter than one mile?  By how much?  
%\end{prob}
%
%\begin{prob}
%Felicia and Wesley are neighbors.  The common boundary between their properties consists of two line segments, as shown below.  
%$$\includegraphics[scale=0.38]{../graphics/TIMSS}$$
%They would prefer their common boundary to be a single straight segment.  How might they change their boundary so that they 
%each have the same area as they have now?  
%\end{prob}
%
%\begin{prob}
%Write an equation of the line through $(2,4)$ parallel to $5x-3y=1$.  
%Now write an equation of the line through $(x_1,y_1)$ parallel to $ax+by=c$. 
%\end{prob}
%
%\begin{prob}
%Write an equation of the line through $(2,4)$ perpendicular to $5x-3y=1$.  
%Now write an equation of the line through $(x_1,y_1)$ perpendicular to $ax+by=c$. 
%\end{prob}

%\begin{prob}
%Intersections of lines.  
%\begin{enumerate}
%\item Find the intersection of the lines $2x-3y=4$ and $3x-5y=3$.  
%\item Find the intersection of the lines $2x-3y=4$ and $-4x+6y=-8$.
%\item Find the intersection of the lines $2x-3y=4$ and $-4x+6y=5$.
%\item How might you have predicted in advance how many solutions to expect for each previous system of equations?
%\item Use algebra to help explain why lines intersect in zero, one, or infinitely many points.  (You know this geometrically, of course.  Here you demonstrate how algebra gives the same result.)  Indicate clearly the algebraic conditions
%for which you get zero, one, or infinitely many points.  
%\end{enumerate}
%\end{prob}

%\begin{prob}
%Suppose you have a rectangle with vertices at $(0,0)$, $(a,0)$,
%$(a,b)$ and $(0,b)$. Use algebra to prove that the diagonals have the
%same length.
%\end{prob}

%\begin{prob}
%Use a general (non-special) triangle to explain why every triangle is a rep-4-tile.  
%Use the same triangle to explain why every triangle tessellates the plane.  Then use your tessellation to explain why every triangle is a rep-$n^2$-tile for any positive integer $n$. 
%\end{prob}
%
%\begin{prob}
%Explain how the ASA congruence criterion follows from the definition of congruence in terms of rigid motions. Be sure to indicate, using the two given angles and the included side, why the sequence of rigid motions guarantees triangle congruence.
%\end{prob}
%

%\begin{prob}
%Find the intersection of the lines
%\begin{align*}
%x_1(t) &= -6 + 9t & x_2(t) &= 3+t \\
%y_1(t) &= 3-2t &  y_2(t) &= -4-2t 
%\end{align*}
%If $(x_1(t),y_1(t))$ gives the position of $\mathrm{jogger}_1$ and
%$(x_2(t),y_2(t))$ gives the position of $\mathrm{jogger}_2$, what is
%the significance of the point of intersection of these lines, from the
%perspective of the joggers?
%\end{prob}
%
%\begin{prob}
%A bug moves according to the following parametric equations, where t is measured in seconds and $x$ and $y$ are measured in centimeters:  $x = 2t^2$, $y = t-2$.  (Suppose $t$ can be any real number.)   
%\begin{enumerate}
%\item Describe the path of the bug.  
%\item Is the bug's position a function of time?  
%\item On the path, is $y$ a function of $x$?  
%\item Is $x$ a function of $y$?  
%\item If you know one of $x$, $y$, or $t$, can you determine the other two?  How does this question relate to the previous two questions?  
%\item In school mathematics, students are often given a graph and asked, ``Is it a function.''  Explain why this is a poor question.  What better questions could you ask?  
%\end{enumerate}
%\end{prob}

%%\subsection{Absolute Value, Distance, and City Geometry}
%\begin{prob}
%Consider the following equations:  
%\setlength{\arraycolsep}{12pt}
%\setlength{\extrarowheight}{3pt}
%\[
%\begin{array}{cccc}
%x^2-y^2=0    &   x^2=y^2   &   |y|=|x|   &   y= \pm x \\
%(x-y)(x+y)=0  &   x= \pm y   &   y = \pm|x|   & x = \pm|y|
%\end{array}
%\]
%\begin{enumerate}
%\item Which equations are equivalent to which other equations?  Say how you know.  (Be sure to state what it means for the equations to be equivalent.)
%\item For each set of equivalent equations, graph the solution set, and describe how each of the equations provides a different way about thinking about that solution set.  
%\end{enumerate}
%\end{prob}
