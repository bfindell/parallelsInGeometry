
5.G.3-4.  Scale back to something less than the full hierarchy.  Begin using definitions. 

7.G.2.  Augment with a clear statement that corresponding sides of similar figures are proportional.  

7.SP.1-4.  Scale back sampling.  
7.SP.5-8.  Scale back probabilities of compound events.  

8.EE.1.  Include a statement that the meanings of zero and negative integer exponents follow from the rules for positive exponents.  (See N-RN.1 for rational exponents.)  

8.G.1-4.  Make clear the importance of transformations without coordinates (8.G.3). 

N-Q.1-3.  These are modeling standards that are more like practices.  

A-SSE.1-3.  These are more like practices.  

A-SSE.4 (geometric series) is an outlier in this domain, and it probably should be a (+) standard.  Make it (+) in A-APR?  

A-APR.2.  Clarify the factor theorem here and the use of the zero product property.  

A-APR.4.  ``Prove polynomial identities'' seems advanced and esoteric.  Connect back to equivalent expressions.  

A-APR.6.  Give a clearer sense what is meant by ``simple rational expressions.''

A-CED.1-4.  These are more like problem-solving practices. 

A-REI.1.  More like a practice standard.  Explain each step in solving a simple equation.  

A-REI.5.  Prove that ... replacing one equation by the sum of that equation ... .   Too subtle for students to prove.  Also, they do it only once.  Omit.  

F-IF.4-5.  Interpret in context.  More like practice standards.  

F-IF.7e.  Logarithmic functions should be clearly (+) standards.  

F-IF.8.  Rewrite to reveal features.  More like practice standards. 

F-BF.4.  Inverse functions.  Improve clarity about what of this is expected of all students.  

G-CO.1.  Intended definitions.  Need more detail about acceptable definitions for G-CO.1.  And to define angle, “ray” would help.  Do we also intend to define the measure of an angle?  Maybe that is why we need distance along a circular arc.  For segment (and for ray), we need some notion of betweenness, which we might define via distance.  If we are not taking an axiomatic approach, do we need to insist that point, line, etc., are formally taken as undefined?  In analytic geometry, after all, point and line are defined.

G-CO.3.  Clarify:  A symmetry of a figure is a transformation that takes the figure onto itself, so that the figure is congruent to itself. 

G.CO-4.  Definitions of basic rigid motions.  The progression says "students learn" the definitions of basic rigid motions, but G-CO.4 says they "develop" them.  Some of my preservice middle-grades teachers struggle to "know" definitions of the special quadrilaterals.  Can we suggest that students should be responsible not for the definitions but rather for properties that follow from the definitions, such as:  

G-CO.5-7.  Improve focus by consider these to be a single standard (analogous to G-SRT.2) that is only a modest step beyond 8.G.1-2.  Also note here the importance of orientation in predicting transformations.  

G-CO.8.  Establishing the triangle congruence criteria.  Important development that should be part of the curriculum.  But my students have considerable trouble distinguishing between a proof for a specific pair of drawn triangles and a proof that can be seen as applying to any pair of triangles satisfying particular conditions.  Can we give a clearer sense of what of this should students be responsible for?  They do this only once.  

G-SRT.3.  Similar argument to G-CO.8.  

G-SRT.8.  Solving right triangles.  Needs clarification regarding solving right triangles: Finding an unknown angle requires inverse trig functions, which are (+) standards.  If finding angles is expected of all students, I worry about all of the baggage that teachers will bring along:  The phrase “inverse sine” and the negative exponent add confusion that few teachers can resolve.  Please help teachers and assessment writers choose among the following possibilities, when given two sides of a right triangle:  
a. Students are not required to find the non-right angles. 
b. Students use tables, graphs, or guess-and-check strategies to approximate the angles. 
c. Teachers, students, and textbooks use functions like “Arcsin,” which is read as “an angle whose sine is.”  We could later admit that sin-1 is another name for Arcsin and apologize for the confusing notation.  

G-GMD.1 or G-GMD.3.  Area and volume under scaling.  In the Geometric Measurement section, can we promote exploration of area and volume under scaling (the missing standard)?  One algebra connection is that area and volume contexts should be prototypical examples of quadratic and cubic functions.  


HS standards that are only slightly beyond grade 7 or 8: 

A-REI.3. 
A.REI-6. 
F-IF.1-2. 
G-CO.2.
G-CO.5-7
S-ID.1-3,5-7. 
S-IC.1,4. 


Almost all of the standards provide facts, ideas, ways of thinking, and problem-solving strategies that continue to be useful in subsequent grades 
(and even in life).  But a few standards are instead important steps in the logical/mathematical development of an idea, but steps that can be subsequently forgotten without much consequence:  

(meaning, making sense, mental arithmetic, problem solving, interpretation, answer-checking, ...

Place value: decomposing a number into tens and ones.  
Division as how many groups, how many in one group, or unknown factor. 
Area models of fraction multiplication
Rules of counting number exponents undergirding meanings of zero and negative exponents
Rules of integer exponents undergirding meanings of rational exponents

4.NF.1.  Explain why a/b = (na)/(nb).  
4.NF.4.b.  Understand a multiple of a/b as a multiple of 1/b. 
5.NBT.2. 2. Explain patterns in the number of zeros of the product when multiplying a number by powers of 10 ...
5.NF.4.a.  Interpret the product (a/b) × q as a parts of a partition of q into b equal parts ...
5.NF.4.b. Find the area of a rectangle with fractional side lengths ...
5.NF.7. Apply and extend previous understandings of division to divide unit fractions by whole numbers and whole numbers by unit fractions
6.G.1.  Find the area of right triangles, other triangles, special quadrilaterals, and polygons by composing into rectangles or decomposing ...
6.G.2.  Find the volume of a right rectangular prism with fractional edge lengths...
7.NS.2.a.  Understand that multiplication is extended from fractions to rational numbers by requiring that operations continue to
satisfy the properties of operations, particularly the distributive property, leading to products such as (–1)(–1) = 1 and the rules
for multiplying signed numbers...
7.NS.2.d. Convert a rational number to a decimal using long division; know that the decimal form of a rational number terminates in 0s or eventually repeats.  
7.G.4.  ... give an informal derivation of the relationship between the circumference and area of a circle.
8.EE.1. Know and apply the properties of integer exponents to generate equivalent numerical expressions
8.EE.6. Use similar triangles to explain why the slope $m$ ...
8.G.1. Verify experimentally the properties of rotations, reflections, and translations...
8.G.6. Explain a proof of the Pythagorean Theorem and its converse.
N-RN.1. Explain how the definition of the meaning of rational exponents follows from extending the properties of integer exponents to those values, allowing for a notation for radicals in terms of rational exponents.


Show that area formula works for rectangles with fractional side lengths
Show that volume formula works for right rectangular prisms with fractional side lengths
[Fundamental assumption of school mathematics]

Establish triangle congruence and similarity criteria.  
Rational versus irrational numbers? 


Sometimes the explanation is simple, meaningful, and generative (fraction multiplication).
Other times ... 

Knowing something
Knowing something follows from something else
Using the fact that something follows from something else
Explaining how something forllows from something else


